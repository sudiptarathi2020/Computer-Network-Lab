
\documentclass{article}
\usepackage{graphicx}
\usepackage{listings}
\usepackage{float}

\usepackage[margin=1in]{geometry} % Adjust margins here

% Front Page
\newcommand{\frontpage}[6]{%
    \begin{titlepage}
        \centering
        \includegraphics[width=0.3\textwidth]{~/Downloads/ju_logo.png}\par\vspace{1cm}
        \vspace{1cm}
        {\scshape\Large Department of Computer Science and Engineering\par}
        \vspace{1.5cm}
        \vspace{0.5cm}
        {\Large Experiment Name: #4\par}
        \vspace{0.5cm}
        {\Large Experiment No: #5\par}
        \vspace{0.5cm}
        {\Large Date: #6\par}
        \vfill
        Submitted to:\par
        Md. Imdadul Islam\par
        Professor of CSE, Jahangirnagar University\par
        \vspace{0.5cm}
        Submitted by:\par
        Name: #1\par
        Exam Roll: #2\par
        Class Roll: #3\par
        \vspace{1cm}
        Jahangirnagar University,Savar, Dhaka\par
        \vfill
    \end{titlepage}
}

\title{IP Telephony}
\author{Sudipta Singha}
\date{\today}

\begin{document}

\frontpage{Sudipta Singha}{202220}{408}{IP Telephony}{6}{\today}


\section{Objective}
% Objective: (3 sentences)
% Write your objective here.
Objective of this experiment is to implement a small network of IP
telephony. Each telephone will be verified with its content of IP
address and corresponding telephone number. Finally, the network
will be tested by dialing to each other IP phone.
\section{Network Diagram}
% Insert your network diagram image here.
\begin{figure}[H]
    \centering
    \includegraphics[width=0.8\textwidth]{~/Pictures/lab6.png}
    \caption{The network of IP Telephony}
\end{figure}

\section{Procedure}
% Write your procedure here, including all commands.
The project is implemented using Cisco Packet Tracer.The command of the router is written below
\subsection{Router configuration using cli}
\begin{lstlisting}[language=bash]
Router>en
Router#conf t
Router(config)#int fa0/0
Router(config-if)#ip add 192.168.10.1 255.255.255.0
Router(config-if)#no shut
Router(config-if)#exit
Router(config)#ip dhcp pool VOICE
Router(dhcp-config)#network 192.168.10.0 255.255.255.0
Router(dhcp-config)#default-router 192.168.10.1
Router(dhcp-config)#option 150 ip 192.168.10.1
Router(dhcp-config)#exit
Router(config)#telephony-service
Router(config-telephony)#max-dn 5
Router(config-telephony)#max-ephone 20
Router(config-telephony)#ip source-address 192.168.10.1 port 2000
Router(config-telephony)#auto assign 1 to 5
Router(config-telephony)#exit
Router(config)#ephone-dn 1
Router(config-ephone-dn)#number 54001
Router(config-ephone-dn)#exit
Router(config)#ephone-dn 2
Router(config-ephone-dn)#number 54002
Router(config-ephone-dn)#
Router(config)#ephone-dn 3
Router(config-ephone-dn)#number 54003
Router(config-ephone-dn)#exit
Router(config)#ephone-dn 4
Router(config)#number 54004
Router(config-ephone-dn)#exit
Router(config)#
\end{lstlisting}
\subsection{Switch configuration using cli}
\begin{lstlisting}[language=bash]
Switch>en
Switch#conf t
Switch(config)#int range fa0/1-23
Switch(config-if-range)#switchport mode access
Switch(config-if-range)#switchport VOICE vlan 1
Switch(config-if-range)#exit
Switch(config)#int fa0/24
Switch(config-if)#switchport mode trunk
Switch(config-if)#exit
Switch(config)#do wr
\end{lstlisting}
\section{Result}
% Write your result here.
\begin{figure}[H]
    \centering
    \includegraphics[width=0.8\textwidth]{~/Pictures/lab6result1.png}
    \caption{The ip phone is reading dial}
\end{figure}
\begin{figure}[H]
    \centering
    \includegraphics[width=0.8\textwidth]{~/Pictures/lab6result2.png}
    \caption{The receiver accept the call}
\end{figure}
\end{document}

