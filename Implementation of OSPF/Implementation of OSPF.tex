
\documentclass{article}
\usepackage{graphicx}
\usepackage{listings}
\usepackage{float}

\usepackage[margin=1in]{geometry} % Adjust margins here

% Front Page
\newcommand{\frontpage}[6]{%
    \begin{titlepage}
        \centering
        \includegraphics[width=0.3\textwidth]{~/Downloads/ju_logo.png}\par\vspace{1cm}
        \vspace{1cm}
        {\scshape\Large Department of Computer Science and Engineering\par}
        \vspace{1.5cm}
        \vspace{0.5cm}
        {\Large Experiment Name: #4\par}
        \vspace{0.5cm}
        {\Large Experiment No: #5\par}
        \vspace{0.5cm}
        {\Large Date: #6\par}
        \vfill
        Submitted to:\par
        Md. Imdadul Islam\par
        Professor of CSE, Jahangirnagar University\par
        \vspace{0.5cm}
        Submitted by:\par
        Name: #1\par
        Exam Roll: #2\par
        Class Roll: #3\par
        \vspace{1cm}
        Jahangirnagar University,Savar, Dhaka\par
        \vfill
    \end{titlepage}
}

\title{Implementation of OSPF}
\author{Sudipta Singha}
\date{\today}

\begin{document}

\frontpage{Sudipta Singha}{202220}{408}{Implementation of OSPF}{7}{\today}


\section{Objective}
% Objective: (3 sentences)
% Write your objective here.
\section{Network Diagram}
% Insert your network diagram image here.
\begin{figure}[H]
    \centering
    \includegraphics[width=0.8\textwidth]{~/Pictures/lab7.png}
    \caption{Network Diagram}
\end{figure}

\section{Procedure}
% Write your procedure here, including all commands.
The network is configured using routers and pcs.The router commands are written below.
\subsection{Router 1}
\begin{lstlisting}[language=bash]
Router>en
Router#conf t
Router(config)#int fa 0/0
Router(config-if)#ip ad 192.168.1.2 255.255.255.0
Router(config-if)#no shut
Router(config-if)#exit
Router(config)# int se0/1/0
Router(config-if)#ip add 192.168.6.1 255.255.255.0
Router(config-if)#no shut
Router(config-if)#exit
Router(config)# int se0/1/1
Router(config-if)#ip add 192.168.8.2 255.255.255.0
Router(config-if)#clock rate 64000
Router(config-if)#no shut
Router(config-if)#exit
Router(config)#exit
Router#
Router#copy running-config startup-config
Router#
Router#conf t
Router(config)#router ospf 1
Router(config-router)#network 192.168.1.0 0.0.0.255 area 0
Router(config-router)#network 192.168.6.0 0.0.0.255 area 0
Router(config-router)#network 192.168.8.0 0.0.0.255 area 0
Router(config-router)#
\end{lstlisting}
\subsection{Router 2}
\begin{lstlisting}[language=bash]
Router>en
Router#conf t
Router(config)#int fa 0/0
Router(config-if)#ip add 192.168.2.2 255.255.255.0
Router(config-if)#no shut
Router(config-if)#exit
Router(config)#int se0/1/0
Router(config-if)#ip add 192.168.7.1 255.255.255.0
Router(config-if)#no shut
Router(config-if)#exit
Router(config)# int se0/1/1
Router(config-if)#ip add 192.168.6.2 255.255.255.0
Router(config-if)#clock rate 64000
Router(config-if)#no shut
Router(config-if)#
Router(config-if)#exit
Router(config)#exit
Router#copy running-config startup-config
Router#conf t
Router(config)#router ospf 1
Router(config-router)#network 192.168.2.0 0.0.0.255 area 0
Router(config-router)#network 192.168.7.0 0.0.0.255 area 0
Router(config-router)#network 192.168.6.0 0.0.0.255 area 0
Router(config-router)#
\end{lstlisting}
\subsection{Router 3}
\begin{lstlisting}[language=bash]
Router>en
Router#conf t
Router(config)#int fa 0/0
Router(config-if)#ip add 192.168.3.2 255.255.255.0
Router(config-if)#no shut
Router(config-if)#exit
Router(config)# int se0/1/0
Router(config-if)#ip add 192.168.7.2 255.255.255.0
Router(config-if)#clock rate 64000
Router(config-if)#no shut
Router(config-if)#exit
Router(config)# int se0/1/1
Router(config-if)#ip add 192.168.8.1 255.255.255.0
Router(config-if)#no shut
Router(config-if)#exit
Router(config)#exit
Router#copy running-config startup-config
Router#
Router#conf t
Router(config)#router ospf 1
Router(config-router)#network 192.168.3.0 0.0.0.255 area 0
Router(config-router)#network 192.168.8.0 0.0.0.255 area 0
Router(config-router)#network 192.168.7.0 0.0.0.255 area 0
Router(config-router)#
\end{lstlisting}

\section{Result}
% Write your result here.
\begin{figure}[H]
    \centering
    \includegraphics[width=0.8\textwidth]{~/Pictures/lab7r1.png}
    \caption{Sending ICMP packet from the source pc}
\end{figure}
\begin{figure}[H]
    \centering
    \includegraphics[width=0.8\textwidth]{~/Pictures/lab7r2.png}
    \caption{The ICMP packet reached the destination}
\end{figure}
\begin{figure}[H]
    \centering
    \includegraphics[width=0.8\textwidth]{~/Pictures/lab7r3.png}
    \caption{Checked using ping command}
\end{figure}
\end{document}

