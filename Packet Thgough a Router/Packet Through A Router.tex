
\documentclass{article}
\usepackage{graphicx}
\usepackage{listings}
\usepackage{float}

\usepackage[margin=1in]{geometry} % Adjust margins here

% Front Page
\newcommand{\frontpage}[6]{%
    \begin{titlepage}
        \centering
        \includegraphics[width=0.3\textwidth]{~/Downloads/ju_logo.png}\par\vspace{1cm}
        \vspace{1cm}
        {\scshape\Large Department of Computer Science and Engineering\par}
        \vspace{1.5cm}
        \vspace{0.5cm}
        {\Large Experiment Name: #4\par}
        \vspace{0.5cm}
        {\Large Experiment No: #5\par}
        \vspace{0.5cm}
        {\Large Date: #6\par}
        \vfill
        Submitted to:\par
        Md. Imdadul Islam\par
        Professor of CSE, Jahangirnagar University\par
        \vspace{0.5cm}
        Submitted by:\par
        Name: #1\par
        Exam Roll: #2\par
        Class Roll: #3\par
        \vspace{1cm}
        Jahangirnagar University,Savar, Dhaka\par
        \vfill
    \end{titlepage}
}

\title{Packet Through A Router}
\author{Sudipta Singha}
\date{\today}

\begin{document}

\frontpage{Sudipta Singha}{202220}{408}{Packet Through A Router}{2}{\today}


\section{Objective}
% Objective: (3 sentences)
% Write your objective here.
In this lab report, we explore how data packets travel through a router. We'll study how routers decide where to send packets and how fast they can do it. By doing this, we aim to understand better how routers help computers communicate on networks.
\section{Network Diagram}
% Insert your network diagram image here.
\begin{figure}[H]
    \centering
    \includegraphics[width=1\textwidth]{~/Pictures/lab2.png}
    \caption{Network Diagram for the Experiment Packet Through A Router}
\end{figure}

\section{Procedure}
% Write your procedure here, including all commands.
We use Cisco Packet Tracer for implementing this experiment.We take all the pc from the cisco packet device
pan.We also take the Router from the device pan.The router we use is Router-PT.We connect all the device using
fastethernet port.We also take 2 switch.Then we assign the ip address and subnet musk for all the computer.We
assign 2 network ip address for the router.We have to insert ehternet card to the router.To test the
connection we send ICMP packet from one node to another node.We also used those commands to test the
connectivity of the network.
\begin{lstlisting}[language=bash]
$ ping 192.168.3.3
$ tracert 192.168.3.3
\end{lstlisting}

\section{Result}
% Write your result here.
\begin{figure}[H]
    \centering
    \includegraphics[width=1\textwidth]{~/Pictures/lab2result1.png}
    \caption{The Packet is leaving the source computer}
\end{figure}
\begin{figure}[H]
    \centering
    \includegraphics[width=1\textwidth]{~/Pictures/lab2result2.png}
    \caption{The Packet reached the router}
\end{figure}
\begin{figure}[H]
    \centering
    \includegraphics[width=1\textwidth]{~/Pictures/lab2result3.png}
    \caption{The Packet reached the destination}
\end{figure}
\begin{figure}[H]
    \centering
    \includegraphics[width=1\textwidth]{~/Pictures/lab2result4.png}
    \caption{The Ackowledgement reached the source node}
\end{figure}
\begin{figure}[H]
    \centering
    \includegraphics[width=1\textwidth]{~/Pictures/lab2result5.png}
    \caption{ping and tracert result}
\end{figure}

\end{document}

