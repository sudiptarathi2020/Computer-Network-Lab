
\documentclass{article}
\usepackage{graphicx}
\usepackage{listings}
\usepackage{float}

\usepackage[margin=1in]{geometry} % Adjust margins here

% Front Page
\newcommand{\frontpage}[6]{%
    \begin{titlepage}
        \centering
        \includegraphics[width=0.3\textwidth]{~/Downloads/ju_logo.png}\par\vspace{1cm}
        \vspace{1cm}
        {\scshape\Large Department of Computer Science and Engineering\par}
        \vspace{1.5cm}
        \vspace{0.5cm}
        {\Large Experiment Name: #4\par}
        \vspace{0.5cm}
        {\Large Experiment No: #5\par}
        \vspace{0.5cm}
        {\Large Date: #6\par}
        \vfill
        Submitted to:\par
        Md. Imdadul Islam\par
        Professor of CSE, Jahangirnagar University\par
        \vspace{0.5cm}
        Submitted by:\par
        Name: #1\par
        Exam Roll: #2\par
        Class Roll: #3\par
        \vspace{1cm}
        Jahangirnagar University,Savar, Dhaka\par
        \vfill
    \end{titlepage}
}

\title{Router Configuration using CLI}
\author{Sudipta Singha}
\date{\today}

\begin{document}

\frontpage{Sudipta Singha}{202220}{408}{Router Configuration using CLI}{4}{\today}


\section{Objective}
% Objective: (3 sentences)
% Write your objective here.
In this experiment we are going learn about router configuration using command line interface (CLI). We will connect routers with pc from command prompt and check their connection through cmd.
\section{Network Diagram}
% Insert your network diagram image here.
\begin{figure}[H]
    \centering
    \includegraphics[width=1\textwidth]{~/Pictures/lab3.png}
    \caption{Network Diagram for Router configuration using CLI}
\end{figure}

\section{Procedure}
% Write your procedure here, including all commands.
The above network diagram is created using Cisco Packet Tracer software. We use 4 PC-PT and 2 Router-PT
here.The configuration of the router done using CLI.\par
\subsection{Router 1}
\begin{lstlisting}[language=bash]
Continue with configuration dialog? [yes/no]: no
Router>en %enable
Router#conf t %configure terminal
Router(config)#int fa0/0 %interface fa0/0
Router(config-if)#ip add 192.168.1.2 255.255.255.0
Router(config-if)#no shut %no shutdown i.e. make the port on
Router(config-if)#
Router(config-if)#exit
Router(config)#int fa1/0
Router(config-if)#ip add 192.168.3.1 255.255.255.0
Router(config-if)#no shut
Router(config-if)#exit
Router(config)#exit
Router#
Router#en
Router#conf t
Router(config)#ip route 192.168.2.0 255.255.255.0 192.168.3.2
Router(config)#end
Router#
Router#sh ip route
C 192.168.1.0/24 is directly connected, FastEthernet0/0
S 192.168.2.0/24 [1/0] via 192.168.3.2
C 192.168.3.0/24 is directly connected, FastEthernet1/0
Router#
\end{lstlisting}
\subsection{Router 2}
\begin{lstlisting}[language=bash]
Continue with configuration dialog? [yes/no]: no
Router>en
Router#conf t
Router(config)#int fa1/0
Router(config-if)#ip add 192.168.2.2 255.255.255.0
Router(config-if)#no shut
Router(config-if)#
Router(config-if)#exit
Router(config)#int fa0/0
Router(config-if)#ip add 192.168.3.2 255.255.255.0
Router(config-if)#no shut
Router(config-if)#exit
Router(config)#exit
Router#
Router#en
Router#conf t
Router(config)#ip route 192.168.1.0 255.255.255.0 192.168.3.1
Router(config)#end
Router#
Router#sh ip route
S 192.168.1.0/24 [1/0] via 192.168.3.1
C 192.168.2.0/24 is directly connected, FastEthernet1/0
C 192.168.3.0/24 is directly connected, FastEthernet0/0
Router#
\end{lstlisting}

\section{Result}
% Write your result here.
\begin{figure}[H]
    \centering
    \includegraphics[width=1\textwidth]{~/Pictures/lab4result1.png}
    \caption{The Packet is leaving the source computer}
\end{figure}
\begin{figure}[H]
    \centering
    \includegraphics[width=1\textwidth]{~/Pictures/lab4result2.png}
    \caption{The Packet reached the first router}
\end{figure}
\begin{figure}[H]
    \centering
    \includegraphics[width=1\textwidth]{~/Pictures/lab4result3.png}
    \caption{The Packet reached the second router}
\end{figure}
\begin{figure}[H]
    \centering
    \includegraphics[width=1\textwidth]{~/Pictures/lab4result4.png}
    \caption{The Packet reached the destination}
\end{figure}
\begin{figure}[H]
    \centering
    \includegraphics[width=1\textwidth]{~/Pictures/lab4result5.png}
    \caption{The ACK reached the source node}
\end{figure}
\begin{figure}[H]
    \centering
    \includegraphics[width=1\textwidth]{~/Pictures/lab4result6.png}
    \caption{Ping result for the given network}
\end{figure}

\end{document}

