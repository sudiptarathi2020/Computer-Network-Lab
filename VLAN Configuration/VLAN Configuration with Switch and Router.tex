
\documentclass{article}
\usepackage{graphicx}
\usepackage{listings}
\usepackage{float}

\usepackage[margin=1in]{geometry} % Adjust margins here

% Front Page
\newcommand{\frontpage}[6]{%
    \begin{titlepage}
        \centering
        \includegraphics[width=0.3\textwidth]{~/Downloads/ju_logo.png}\par\vspace{1cm}
        \vspace{1cm}
        {\scshape\Large Department of Computer Science and Engineering\par}
        \vspace{1.5cm}
        \vspace{0.5cm}
        {\Large Experiment Name: #4\par}
        \vspace{0.5cm}
        {\Large Experiment No: #5\par}
        \vspace{0.5cm}
        {\Large Date: #6\par}
        \vfill
        Submitted to:\par
        Md. Imdadul Islam\par
        Professor of CSE, Jahangirnagar University\par
        \vspace{0.5cm}
        Submitted by:\par
        Name: #1\par
        Exam Roll: #2\par
        Class Roll: #3\par
        \vspace{1cm}
        Jahangirnagar University,Savar, Dhaka\par
        \vfill
    \end{titlepage}
}

\title{VLAN Configuration with Switch and Router}
\author{Sudipta Singha}
\date{\today}

\begin{document}

\frontpage{Sudipta Singha}{202220}{408}{VLAN Configuration with Switch and Router}{5}{\today}


\section{Objective}
% Objective: (3 sentences)
% Write your objective here.
The objective of this experiment is to implement VLAN using switch and router. Then we will simulate it in packet tracer. For real life simulation we will taste and trace it in command prompt.
\section{Network Diagram}
% Insert your network diagram image here.
\begin{figure}[H]
    \centering
    \includegraphics[width=1\textwidth]{~/Pictures/lab5_1.png}
    \caption{VLAN using A switch}
\end{figure}

\begin{figure}[H]
    \centering
    \includegraphics[width=1\textwidth]{~/Pictures/lab5(1).png}
    \caption{VLAN under sub-interface}
\end{figure}

\section{Procedure}
% Write your procedure here, including all commands.
We take pc, switch and router from the cisco packet tracer device pan.The commands for router is written
below.
\subsection{Router Configuration using CLI}
\begin{lstlisting}[language=bash]
Switch>en
Switch#conf t
Switch(config)#vlan 10
Switch(config-vlan)#name A
Switch(config-vlan)#exit
Switch(config)#vlan 20
Switch(config-vlan)#name B
Switch(config-vlan)#exit
Switch(config)#vlan 30
Switch(config-vlan)#name C
Switch(config-vlan)#exit
Switch(config)#int range fa0/1-6
Switch(config-if-range)#switchport access vlan 10
Switch(config-if-range)#exit
Switch(config)#int range fa0/7-14
Switch(config-if-range)#switchport access vlan 20
Switch(config-if-range)#exit
Switch(config)#int range fa0/15-23
Switch(config-if-range)#switchport access vlan 30
Switch(config-if-range)#exit
Switch(config)#int fa0/24
Switch(config-if)#switchport mode trunk
Switch(config-if)#end
Router>en
Router#conf t
Router(config)#int fa0/0
Router(config-if)#no shut
Router(config-if)#int fa0/0.1
Router(config-subif)#encapsulation dot1q 1
Router(config-subif)#ip add 192.168.1.1 255.255.255.0
Router(config-subif)#int fa0/0.2
Router(config-subif)#encapsulation dot1q 10
Router(config-subif)#ip add 192.168.10.1 255.255.255.0
Router(config-subif)#int fa0/0.3
Router(config-subif)#encapsulation dot1q 20
Router(config-subif)#ip add 192.168.20.1 255.255.255.0
Router(config-subif)#int fa0/0.4
Router(config-subif)#encapsulation dot1q 30
Router(config-subif)#ip add 192.168.30.1 255.255.255.0
Router(config-subif)#end
\end{lstlisting}

\section{Result}
% Write your result here.
\begin{figure}[H]
    \centering
    \includegraphics[width=1\textwidth]{~/Pictures/lab5result1.png}
    \caption{The packets will not reach the destination in this Configuration}
\end{figure}
\begin{figure}[H]
    \centering
    \includegraphics[width=1\textwidth]{~/Pictures/lab5result2.png}
    \caption{The packet is now leaving the source pc}
\end{figure}
\begin{figure}[H]
    \centering
    \includegraphics[width=1\textwidth]{~/Pictures/lab5result3.png}
    \caption{The ICMP packet reached the router}
\end{figure}
\begin{figure}[H]
    \centering
    \includegraphics[width=1\textwidth]{~/Pictures/lab5result4.png}
    \caption{The ICMP packet reached the destination node}
\end{figure}
\begin{figure}[H]
    \centering
    \includegraphics[width=1\textwidth]{~/Pictures/lab5result5.png}
    \caption{The ACK reached the source}
\end{figure}
\begin{figure}[H]
    \centering
    \includegraphics[width=1\textwidth]{~/Pictures/lab5result6.png}
    \caption{Ping is successful}
\end{figure}

\end{document}

